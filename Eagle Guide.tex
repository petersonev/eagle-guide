\documentclass{article}

\usepackage{geometry}
\geometry{a4paper, portrait, margin=1.5in}

\usepackage{hyperref}
\hypersetup{
    colorlinks=true,
    linkcolor=black,
    filecolor=magenta,
    urlcolor=blue,
}

\usepackage{tcolorbox}
\usepackage{textcomp}
\usepackage{gensymb}

\title{Autodesk Eagle CAD Guide}
\author{Evan Peterson}
\begin{document}
\maketitle{}
\setcounter{tocdepth}{2}
\tableofcontents
\pagebreak

\section{Introduction}
Eagle is a CAD tool used to design printed circuit boards (PCBs). The program is
split up into three sections which build upon one another to result in files
that are sent to a PCB manufacturer.\\
First all parts to go on the board are created in libraries which describe the
pins on the part and the footprint of the part for when it is soldered to the
board. Then these parts are used in the Schematic design which describes the
electrical connections on the board. Then the board is layed out by following
the parts and connections made in the schematic to place parts on the physical
board then route where the physical connections will go between parts.
\begin{tcolorbox} [title=Tips \& Tricks]
    \begin{itemize}
        \item Everything in Eagle can be done through clicking buttons or by
        using its command line. Design can be done much quicker by learning to
        use the command line.
        \item In order to use some Eagle functions like creating new files or
        adjusting some settings you must switch back to Eagle's Control Panel
        window
    \end{itemize}
\end{tcolorbox}

\subsection{Installation}
\begin{enumerate}
    \item Download at \url{www.autodesk.com/products/eagle}
    \item Install Eagle using free or educational version
    \begin{itemize}
        \item Free version: 2 Board Layers, 80cm\textsuperscript{2} Board Area
        \item Educational version: 16 Board Layers, Unlimited Board Area
        \begin{itemize}
            \item \url{www.autodesk.com/education/free-software/eagle}
        \end{itemize}
    \end{itemize}
\end{enumerate}

\section{Libraries}
Libraries are used to store information about a part which are used in the
schematic and the board layout. Eagle has built in libraries for many common
parts but very often you have to make your own.\\
Libraries are split into three parts. The Symbol is what is seen in the
schematic and is how you describe the pins on your part which electrical
connections go to. The Package is the footprint used for board layout which you
will solder the component to on the physical board. The Device brings the Symbol
and Package together so what is created in the schematic can be translated
into the board layout.

\subsection{Create Library} \label{create library}
\begin{enumerate}
    \item (At Control Panel) \textit{\textbf{File\textrightarrow
    New\textrightarrow Library}}
    \item Save to a directory where you will keep all Eagle libraries
\end{enumerate}
\begin{tcolorbox} [title=Tips \& Tricks]
    \begin{itemize}
        \item Libraries are commonly split up into components types such as
        resistors or connectors
        \item Eagle has a default folder it searches for its libraries but you
        can have it search additional folders for your libraries by (At Control
        Panel) \textit{\textbf{File\textrightarrow Options\textrightarrow
        Directories}} then appending to the Libraries path a colon then the path
        to your directory
    \end{itemize}
\end{tcolorbox}

\subsection{Library Part: Symbol}
\begin{enumerate}
    \item Create new symbol: \textit{\textbf{Library\textrightarrow Symbol}}
    \item Name symbol with part number
    \item Draw box using \textit{\textbf{wire}}
    \begin{itemize}
        \item This aspect is purely visual so any size/shape is okay but use
        standard symbols which are commonly represented in schematics
    \end{itemize}
    \item Add neccasary amount of pins using \textit{\textbf{pin}}
    \begin{itemize}
        \item These pins are what wires are connected to in the schematic
    \end{itemize}
    \item Name pins corresponding to datasheet using \textit{\textbf{name}} then
    click on pin
    \item Clean up Symbol: Adjust part outline so all pins fit and pin names can
    be seen
    \item Add Name and Value using \textit{\textbf{text}}
    \begin{itemize}
        \item \textgreater NAME and  \textgreater VALUE are keywords for text
        that are automatically filled out with the name and value when inserted
        into a schematic.
    \end{itemize}
    \item Change Layer of Name and Value using \textit{\textbf{info}} then
    clicking on the part and changing Layer to Name and Value respectivly
    \item Return to Library: \textit{\textbf{Library\textrightarrow Table
    of Contents}}
\end{enumerate}
\begin{tcolorbox} [title=Tips \& Tricks]
    \begin{itemize}
        \item Always keep pins snapped to the 0.1 in grid
        \item Right click while placing or moving to rotate
        \item To move a group use \textit{\textbf{group}}, select what you want
        to move then \textit{\textbf{Right Click\textrightarrow Move:Group}}
        \item Objects are clickable and selectable from the grey +
    \end{itemize}
\end{tcolorbox}

\subsection{Library Part: Package}
This is typically the footprint for the package which is found in the datasheet
for the part. In the datasheet you will find all measurements needed to create
the package. This package will go on the PCB and is what the component will be
soldered to.
\begin{enumerate}
    \item Create new package: \textit{\textbf{Library\textrightarrow Package}}
    \item Name package with package name
    \begin{itemize}
        \item Common for multiple parts to have the same package
    \end{itemize}
    \item Place first pad
    \begin{itemize}
        \item Place surface mount using \textit{\textbf{SMD}}
        \item Place through hole pad using \textit{\textbf{pad}}
    \end{itemize}
    \item Resize pad using \textit{\textbf{info}} clicking on pad then changing
    SMD size to size given in datasheet
    \item Use \textit{\textbf{copy}} to copy correct amount of pads and place
    them in approximate positions
    \item Use values given in datasheet to calculate position (to pad center)
    for each pad then move pads using \textit{\textbf{info}} and change
    position values
    \item Rename pads using \textit{\textbf{rename}}. Pad numbers given by
    datasheet
    \item Add part outline using \textit{\textbf{line}} then changing Layer
    dropdown to \textit{\textbf{51 tDocu}} then draw basic part outline
    \begin{itemize}
        \item The tDocu is just for reference to ensure no overlapping parts so
        it will not appear on the board.
    \end{itemize}
    \item Adjust part outline to the dimmensions given on the datasheet. Adjust
    to exact values using \textit{\textbf{info}}
    \item Add silkscreen
    \begin{itemize}
        \item Silkscreen gets printed on the board and is used to help place
        components
        \item Draw lines on the \textit{\textbf{21 tPlace}} layer as an outline
        that does not intersect with pads
        \item Add a dot to indicate where Pin 1 is on component to help when
        soldering component
    \end{itemize}
    \item Add Name and Value using \textit{\textbf{text}}
    \begin{itemize}
        \item Use \textgreater NAME and  \textgreater VALUE similar to in symbol
        \item Adjust Layer to Name and Value respectivly
    \end{itemize}
    \item Return to Library: \textit{\textbf{Library\textrightarrow Table
    of Contents}}
\end{enumerate}
\begin{tcolorbox} [title=Tips \& Tricks]
    \begin{itemize}
        \item Change grid size using \textit{\textbf{grid}}
        \begin{itemize}
            \item Change grid size to whatever is helpful to you
            \item Use same units as given in datasheet. Values for position and
            size are in these units
            \item Reccomended: Size = 0.5mm, Alt = 0.25mm
        \end{itemize}
        \item Use positions such that (0,0) is at the center of the component
    \end{itemize}
\end{tcolorbox}

\subsection{Library Part: Device}
This is where the Symbol and Package come together to create what will be
added to the schematic.
\begin{enumerate}
    \item Create new device: \textit{\textbf{Library\textrightarrow Device}}
    \item Name with general part number
    \begin{itemize}
        \item Not package specific because a single device can have multiple
        packages
    \end{itemize}
    \item Add symbol using \textit{\textbf{add}} then select previously created
    symbol
    \item Place symbol so grey + is in center of component
    \item Click new button in package window and select previously created
    package
    \item Connect pins by clicking connect button in package area
    \item Use datasheet to see what pin names correspond to what pad numbers
    then connect by selecting pin and pad then click connect. Repeat for each
    one
    \item Click prefix button to add corrent prefix to match component type
\end{enumerate}
\begin{tcolorbox} [title=Tips \& Tricks]
    \begin{itemize}
        \item A wildcard character \textit{\textbf{*}} can be used in Device
        name to allow you have names for different versions of the part without
        creating a new device. Example: You have parts A45C and A46C that have
        the same pins and package but have some minor internal difference. You
        can name device A4*C then specify two technologies: 5 and 6.
        \item You can add multiple packages and set the variant name to specify
        the package. This variant name is appended to the end of the Device
        name
    \end{itemize}
\end{tcolorbox}

\begin{itemize}
    \item Done with library part and ready to be added to schematic
    \item Many more parts can be added to this library. Parts that share
    packages only need package to be created once so check if that package
    exists first so you don't have to remake it.
\end{itemize}

\section{Schematic}
Schematics are used to set up how the board is organized by adding all the parts
used on the board and making the electrical connections between these parts.
\begin{enumerate}
    \item (At Control Panel) \textit{\textbf{File\textrightarrow
    New\textrightarrow Schematic}}
    \item Save to a directory where you are storing this project
\end{enumerate}
\begin{tcolorbox} [title=Tips \& Tricks]
    \begin{itemize}
        \item Always keep parts and wires snapped to 0.1 in grid
    \end{itemize}
\end{tcolorbox}

\subsection{Schematic: Adding Parts}
\begin{enumerate}
    \item Prepare Eagle with neccesary libraries
    \begin{itemize}
        \item Libraries can be added by automatically searching a folder as
        shown in section \ref{create library}
        \item Libraries can also be temporarily added by
        \textit{\textbf{Library\textrightarrow Use}} then select library you
        created
    \end{itemize}
    \item Add components using \textit{\textbf{add}} then select the library
    containing the component then select the componenet
    \begin{itemize}
        \item Add a frame to keep schematic organized
        \item Add all components needed for schematic
        \item Resistors and Capacitors are found in rcl library. R-US and C-US
        are the parts commonly used and the SMD package 0603 is commonly used
        (parts named R-US\_R0603 and C-USC0603 respectivly)
    \end{itemize}
    \item Change value of parts using \textit{\textbf{value}}
    \begin{itemize}
        \item This value has no effect on the board, only used as reference
    \end{itemize}
\end{enumerate}
\begin{tcolorbox} [title=Tips \& Tricks]
    \begin{itemize}
        \item When adding components you can search for components but Eagle
        uses exact matching so you must add wild cards. Example: you want to
        search for part number containing 0603, search using *0603*
    \end{itemize}
\end{tcolorbox}

\subsection{Schematic: Wiring}
\begin{enumerate}
    \item Organize components
    \begin{itemize}
        \item Components need to be placed in a way so it is easy to read
    \end{itemize}
    \item Use the net tool draw connections between components
    \begin{itemize}
        \item NOT the wire tool. The wire tool is only used for visuals, does
        not make any actual connections
    \end{itemize}
    \item Connect power and ground using parts from the supply library. All
    supply parts that are the same on a schematic are connected together
    \begin{itemize}
        \item Good practice to have ground supply parts pointing down and power
        supply parts pointint up
    \end{itemize}
\end{enumerate}
\begin{tcolorbox} [title=Tips \& Tricks]
    \begin{itemize}
        \item Connections can be made without running nets between components.
        This is done by using \textit{\textbf{name}} on a multiple nets and
        naming them the same thing. All nets with the same name are treated as
        being connected.
        \begin{itemize}
            \item Pins cannot be named, add a short net connecting to the pin
            then name that net
            \item Name these nets something useful so it is easier to follow
            \item To see that nets are named the same and connected you can use
            \textit{\textbf{label}} on a net and place the label at the end of
            the net
        \end{itemize}
        \item Connections can be checked with \textit{\textbf{show}}
    \end{itemize}
\end{tcolorbox}

\subsection{Schematic: Error Checking}
\begin{itemize}
    \item Use the Electrical Rule Check (ERC) to check for electrical errors
    and warnings
    \item The ERC is often useful to show connections you missed
    \item Many warnings can be ignored from the ERC because they often have to
    do with net names, etc.
\end{itemize}

\section{Board Layout}
The board file is how you layout the final physical board. All steps up to this
point have been to assist you in the creation of this.
\begin{enumerate}
    \item When done with schematic switch to board by
    \textit{\textbf{File\textrightarrow Use}}
    \begin{itemize}
        \item If no board file found one will be generated
        \item It is generated with parts from the schematic randomply placed and
        lines showing the connections to be made between parts called airwires
    \end{itemize}
\end{enumerate}

\subsection{Board Layout: Layers}
\begin{itemize}
    \item What is actually printed:
    \begin{itemize}
        \item Top Silkscreen (printed on top of board for reference, typically
        white)
        \item Top Soldermask (applied over copper to protect from solder,
        typically green)
        \item Top Copper (makes electrical connections and components are
        soldered to)
        \item Substrate (FR4: used to support and seperate copper layers)
        \item Bottom Copper
        \item Bottom Soldermask
        \item Bottom Soldermask
    \end{itemize}
    \item Use \textit{\textbf{layer}} to adjust what layers are currently viewed
\end{itemize}
\begin{tcolorbox} [title=Tips \& Tricks]
    \begin{itemize}
        \item Right click on Layer settings to create a group of currently
        viewable layers or quickly switch between created groups
    \end{itemize}
\end{tcolorbox}

\subsection{Board Layout: Arranging}
\begin{itemize}
    \item Use \textit{\textbf{move}} and right click to rotate
    \item How you arrange your parts has a large impact on the routing
    routing difficulty in the next step
    \item First consider the requirements of your board
    \begin{itemize}
        \item Consider the maximize size you want your board to be
        \item Consider Location of specific parts
        \begin{itemize}
            \item Specific locations of connectors and other components that
            need to be accessible
            \item Decoupling capacitors very close to its IC
            \item Maximum distances parts can be that must communicate
        \end{itemize}
        \item Consider mounting holes
        \item Consider clearance with other boards or objects around it
    \end{itemize}
    \item Then consider what is easiest placement for when you route the board
    \begin{itemize}
        \item Leave space between parts so no parts collide and so you have
        enough space for routing
        \item Group parts together based on function, refering back to
        schematic. Reduces required routing distances
        \item Minimize intersecting airwires. It is much more difficult to route
        when airwires cross
        \item Reccomended keeping as many components as you can on the top layer
        so you can use the bottom layer for routing or ground plane
    \end{itemize}
    \item Use \textit{\textbf{delete}} to erase existing dimmension then use
    \textit{\textbf{wire}} with the layer set to \textit{\textbf{20 Dimmension}}
    then draw a new dimmension which all your parts are within
\end{itemize}

\subsection{Board Layout: Routing}
\begin{itemize}
    \item Routing is making all the connections shown by the airwires without
    overlapping anything
    \item Two layer board so you can place components and route on the top or
    the bottom
    \begin{itemize}
        \item Vias are used to connect between the top and bottom layers
    \end{itemize}
    \item Don't worry about ground connections yet, a ground plane will be added
    later
    \item Use \textit{\textbf{route}} (not \textit{\textbf{wire}})
    \begin{itemize}
        \item Use layer dropdown to select signal layer you are routing on
        \item Optional selection for walkarount obstacles to assist you with
        routing to avoid overlapping based on DRC rules
        \item Bend style is the angle of wires when routing (good practice to
        use 45\degree\ angles)
        \item Width is how wide the copper trace is (reccomended 0.3mm)
        \begin{itemize}
            \item Trace width must be considered for some applications such
            as power lines
        \end{itemize}
        \item Via settings are for size and shape of via (recomended circle
        with drill of 0.35mm)
    \end{itemize}
    \item Start from end of an air wire and route to the other end of that
    airwire
    \item Left click when routing to place segment and continue routing
    \item Ensure no overlap when routing
    \begin{itemize}
        \item For pads and traces on the same layer there must be zero overlap
        \item Traces on seperate layers can overlap
        \item Vias and through hole components (clored in green) are on both
        layers so traces cannot overlap with these on either layer
    \end{itemize}
    \item Vias can be added by changing layers while routing (using middle
    click) or by adding manually with the via tool
    \begin{itemize}
        \item Vias added manually do not have the net automatically set so must
        use \textit{\textbf{name}} to change name of via to same as net you
        want to connect it to
    \end{itemize}
    \item Different PCB manufacturerers have specifications for the minimum
    distance they can produce between traces
    \begin{itemize}
        \item You must ensure your traces are at least that far apart
    \end{itemize}
    \item Traces and vias cannot be removed with \textit{\textbf{delete}}, you
    must use \textit{\textbf{ripup}}
    % add about bus routing
\end{itemize}
\begin{tcolorbox} [title=Tips \& Tricks]
    \begin{itemize}
        \item Right click while routing to switch bend styles
        \item Use \textit{\textbf{ratsnest}} to recalculate airwires to shortest
        length
    \end{itemize}
\end{tcolorbox}

\subsection{Board Layout: Polygons}
Polygons are how large sections of copper (like ground planes) are done. Using
a ground plane allows you to connect many ground pins without routing as well as
other benefits.
\begin{enumerate}
    \item Use \textit{\textbf{polygon}} to draw the shape around you want to
    fill
    \begin{itemize}
        \item For ground plane you typically want to fill the whole board so
        draw the polygon along the dimmension lines
    \end{itemize}
    \item Use \textit{\textbf{name}} to connect this polygon to the net with the
    same name
    \item Use \textit{\textbf{ratsnest}} to fill in the polygons
\end{enumerate}
\begin{tcolorbox} [title=Tips \& Tricks]
    \begin{itemize}
        \item Change the polygon's rank when you have intersecting polygons to
        give one higher priority
        \item Add polygon inside of another and change Polygon pour to cutout if
        there is a section of the board you don't want a polygon
        \item Check or uncheck thermals based on application (typically good
        idea to use when polygon overlaps pads so the component is easier to
        solder on)
    \end{itemize}
\end{tcolorbox}

\subsection{Board Layout: Silkscreen}
\begin{itemize}
    \item Silkscreen is printed on the top and bottom of the board and has no
    effect on the function of the board. Rather it is used to identify parts on
    the board and to help with component soldering.
    \item Most of the silkscreen comes from individual parts on Names layer
    identifying part names
    \item Often these names for parts are misaligned. Use
    \textit{\textbf{smash}} on the component to allow you to move these labels
    \item Add other markings to silkscreen on layers tPlace or bPlace for the
    top and bottom silkscreen respectivly
\end{itemize}

\subsection{Board Layout: Error Checking}
\begin{itemize}
    \item Use \textit{\textbf{ratsnest}} then check the notification at the
    bottom left for how many airwires are left to see if there is stuff left to
    route. Everything is routed if "Nothing to do"
    \item Use the Design Rule Check (DRC) to check for clearance, overlap, etc
    \item Settings of DRC can be changed to fit the requirements of where your
    board is being printed
    \item When "Ratsnest: Nothing to do" and "DRC: No errors" you are done
\end{itemize}

\subsection{Board Layout: Exporting}
You must export the board into Gerber files which are sent to the manufacturer
to print. This is a set up multiple files, one for each layer
\begin{enumerate}
    \item Use \textit{\textbf{File\textrightarrow CAM Processor}} then
    \textit{\textbf{File\textrightarrow Open\textrightarrow Job}} and select
    gerb274.cam
    \begin{itemize}
        \item There is also a 4 layer version of this CAM
    \end{itemize}
    \item This exports a file for each layer with the exception of of bottom
    silkscreen. This can be added if desired by clicking add and selecting
    Dimmension, bPlace, and bNames
    \item Repeat CAM Processor with excellon.cam job to generate drill file
\end{enumerate}
% \subsection{Layers}
% reference Prefixes
% reference Layers
% reference Commands

% \section{Ordering}
% \section{Soldering}
\end{document}